%----------------------------------------------------------------
\cvsection{经验}

%----------------------------------------------------------------
\begin{cventries}
	\cventry
	{Web前端开发 , 数据科学学习中}
	{自由职业}
	{奥克兰, \enskip 新西兰}
	{4月 2018 - Present}
	{
		\begin{cvitems}
			\item {使用MEAN stack, Socket.io 开发网页版\href{https://github.com/ChenMu1/chatroom}{\color{deepblue}{在线聊天室}}
			\item{使用React, React router, webpack etc. 开发网页版\href{https://github.com/ChenMu1/amazing-react-app}{\color{deepblue}{前端单页管理仪表板}} 
			\item{业余时间学习Python, Machine learning, Data Visualisation \quad  \href{https://www.kaggle.com/chenmu1/kernels}{\color{deepblue} {Kaggle} }
		}
	 }
		}
		\end{cvitems}
	}
\end{cventries}

%----------------------------------------------------------------
\begin{cventries}
	\cventry
	{Web后端开发人员}
	{\href{http://www.aimyplus.com/about}{aimyPlus}}
	{奥克兰, \enskip 新西兰}
	{3月 2017 – 3月 2018}
	{
		\begin{cvitems}
			\item{Agile 团队的环境下维护Fambam项目与开发后端API。 Fambam是 \href{http://www.aimyplus.com}{\color{deepblue}{aimyPlus}}开发的一款儿童日托活动管理软件}
			\item {负责编写相关RESTful风格API, 让其他应用程序来调用,形成生态,让儿童日托活动管理软件发挥最大的价值,从而提高系统的维护性和扩展性。 主用ASP.NET 5}
			\item{使用内建的ASP.NET帮助器,解决发送电子邮件和相关用户验证等问题。了解.NET设计模式的使用场景,虽然没有直接参与软件构架,但对于公司使用的MVC软件模式有自己的见解}
			\item {工作期间,交换到UI 团队,通过与设计师合作,实现相关功能。同时也通过与设计师前辈的学习,横向扩大对设计的理解,认真思考让自己能对应商业的更高层面}
		\end{cvitems}
	}
\end{cventries}

%----------------------------------------------------------------


